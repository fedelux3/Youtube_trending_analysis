\documentclass[10pt, a4paper,openany]{article}
\usepackage[italian]{babel}
\usepackage[T1]{fontenc}
\usepackage[table]{xcolor}
\usepackage{float}
\restylefloat{table,figure}
\usepackage{graphicx}	
\usepackage[utf8]{inputenc}
\usepackage{amsmath}
\usepackage{fancyhdr}
\usepackage{geometry}
\usepackage{url}
\usepackage{hyperref}
\usepackage[ruled,vlined]{algorithm2e}
\geometry{a4paper,top=2cm,bottom=2cm,left=3cm,right=3cm,%
	heightrounded,bindingoffset=5mm}
\usepackage{amssymb}
\usepackage{amsthm}


\begin{document}

\begin{center}
\huge\textbf{YouTube al tempo del Covid-19}

Un'analisi dei video in tendenza
\end{center}

\begin{center}
Gabriele Celeri 847335 (TTC), Federico Luzzi 816753 (DS),\\  Marco Peracchi 800578 (DS), Christian Uccheddu 800428 (DS)
\end{center}

\hrule
\vspace{0.2cm}
\begin{center}\textbf{Introduzione e obiettivi}\end{center} 
Negli ultimi mesi il Covid-19 ha avuto un impatto notevole sulle vite di tutti noi. L'obiettivo di questo lavoro è capire se la piattaforma YouTube sia stata influenzata dalla presenza di questo virus che ha costretto a rimanere nelle proprie case gran parte della popolazione mondiale. 
Per valutare l'impatto della pandemia e della quarantena obbligatoria, abbiamo confrontato i video in tendenza di YouTube nel periodo di dicembre-gennaio rispetto a quello di marzo-maggio, con l'obiettivo di verificare se i contenuti presenti sulla piattaforma e la loro tipologia siano cambiati. Nel nostro lavoro assumiamo che i video in tendenza rappresentino la moda del momento, ovvero quello che le persone cercano e guardano maggiormente.
Inoltre, ci siamo preposti, come secondo obiettivo, di verificare se l'andamento dei contagi e delle notizie in merito, specialmente se negative, influenzassero in alcun modo la fruizione di video riguardanti il Covid-19.
\\\\ \begin{small}
	\textit{Keyword: Covid-19, YouTube}
\end{small}
\vspace{0.2cm}
\hrule

\subsection*{Scelta degli strumenti}
Per rispondere alle nostre domande di ricerca dobbiamo comprendere su quali delle tre "V" della Data Management avremmo dovuto concentrare i nostri sforzi. La conclusione a cui siamo giunti si focalizza sulla \textit{Volume} e \textit{Variety}. Il trattamento di grandi quantità di dati è risultato fondamentale, in quanto tramite le API di Youtube è stato possibile ricavare una discreta quantità di informazioni riguardanti i video in tendenza ($\sim 3Gb$). I dati riguardanti la pandemia sono stati ricavati in formato \textit{csv}, di conseguenza l'integrazione con i dati di Youtube, in formato \textit{json}, è stata fondamentale.

Per la gestione dei dati da Youtube ci siamo affidati al software MongoDB, sul quale sono stati caricati i dati mediante script python. Le API di Youtube non permettevano di scegliere il periodo, ma fornivano i dati in tempo reale, per questo abbiamo utilizzato Apache Kafka per avere i dati salvati per poi poterli caricare su MongoDB.
 

\section*{Raccolta dati}
La raccolta dati è basata su due fonti principali: YouTube e [fonte/i covid-19].

\subsection*{Youtube API}
I video in tendenza su Youtube variano ogni 15 minuti circa (fonte: \url{https://support.google.com/youtube/answer/7239739?hl=it}). Tuttavia questo non significa che cambino effettivamente i contenuti presenti, infatti solitamente si assiste solo a qualche video che scompare dalle tendenze o viceversa. Va specificato inoltre che il numero assoluto di video in tendenza è di circa 200, con qualche calo durante la notte più rilevante.

Le API di Youtube ci hanno permesso di raccogliere i dati necessari riguardanti i video in tendenza, con delle limitazioni sul numero di richieste gratuite che si potessero fare mediante il Google Developer.
\\
Sono state affrontate due sessioni di \textit{scraping} con metodi leggermente differenti:
\begin{enumerate}
	\item dal 23 dicembre 2019 al 5 gennaio 2020 (raccolta ogni 30 minuti)
	\item dal 18 marzo 2020 al 6 maggio 2020 (raccolta ogni 6 ore)
\end{enumerate}
Abbiamo deciso di raccogliere dati delle tendenze dei seguenti paesi:

Italia, USA, Regno Unito, India, Germania, Canada, Francia, Corea del sud, Russia, Giappone, Brasile, Messico\\

\subsection*{Prima sessione}
L'idea che ha guidato la fase iniziale del progetto era quella di comprendere come l'algoritmo delle tendenze di Youtube sceglie i video, utilizzando caratteristiche come visualizzazioni, likes, dislikes e commenti. La raccolta dati ha seguito il seguente algoritmo, messo in pratica dallo script \textit{scraper\_timed.py}:\\
\\
\begin{algorithm}[H]
	\KwData{country - insieme dei paesi prescelti; videos - insieme di video scaricati da un determinato paese}
	\nl \For{every 30 minutes} {
	\nl \ForEach{country}
	{
		\nl videos = APIrequest(max(50 video), country)\\
		\nl \While {tendency videos are finished}
		{
			\nl videos += APIrequest(max(50 video), country) \\
		}
		\nl videos\_fix = arrangeData(videos)\\ 
		\nl saveToCsv(videos\_fix) \\
		\nl videos = $\emptyset$
	}
	}
	\caption{Scraping youtube $\rightarrow$ csv}
\end{algorithm}

%Alternativa per scrivere l'algoritmo
%\begin{enumerate}
%	\item Per ogni paese prescelto
%	\item API request sulle tendenze del momento $\rightarrow$ download di page da 50 video [formato json]
%	\item ripeti 2. finchè scarica tutti i video
%	\item copio le informazioni formattate in formato csv (vedi dopo) 
%	\item ripeti da 2. per ogni paese
%	\item attendi 30 minuti (scheduler)
%	\item ripeti da 1.
%\end{enumerate}
\underline{Nota}: per ogni richiesta API è possibile scaricare un massimo di 50 video, inoltre ogni paese ha un numero di video differente (solitamente 150-200).

I dati vengono salvati in formato \textit{csv} secondo il seguente schema:
\begin{table}[H]
	\centering
	\begin{tabular}{l|l}
		\textbf{Attributo} & \textbf{Descrizione} \\
		\hline
		\textbf{timestamp} & data, ora e minuto della nostra rilevazione \\\hline
		\textbf{video\_id} & identificativo unico del video\\\hline
		\textbf{title} & nome del video per esteso\\\hline
		\textbf{publishedAt} & data di pubblicazione\\\hline
		\textbf{channelId} & identificativo unico del canale che ha pubblicato il video\\\hline
		\textbf{channelTitle} & nome del canale per esteso\\\hline
		\textbf{categoryId} & identificativo unico della categoria\\\hline
		\textbf{trending\_date} & data in cui il video è in tendenza\\\hline
		\textbf{tags} & stringa contenente i tag usati, separati dal carattere "|"\\\hline
		\textbf{view\_count} & numero di visualizzazioni\\\hline
		\textbf{likes} & numero di like (mi piace)\\\hline
		\textbf{dislikes} & numero di dislike (non mi piace)\\\hline
		\textbf{comment\_count} & numero di commenti sotto il video\\\hline
		\textbf{thumbnail\_link} & url all'immagine di copertina del video\\\hline
		\textbf{comments\_disabled} & booleano che dichiara se i commenti sono disabilitati\\\hline
		\textbf{ratings\_disabled} & booleano che dichiara se i like/dislike sono disabilitati\\\hline
		\textbf{description} & descrizione del video\\
		\hline
	\end{tabular}
	\caption{schema degli attributi dei dati csv}
\end{table}

I dati così raccolti sono salvati nella cartella [INSERIRE NOME CARTELLA]

\subsection*{Seconda sessione}

Lo scoppio della pandemia da Covid-19 ci ha permesso di cambiare approccio e domande di ricerca, cercando di valutare come la pandemia abbia influenzato Youtube. Con l'esperienza della presa dati precedente abbiamo cambiato metodo di raccolta, preferendo immagazzinare i dati direttamente, attraverso una pipeline, in un database MongoDB.

Inoltre abbiamo effettuato cambiamenti all'algoritmo di scraping, preferendo effettuare le richieste API ogni sei ore, per un totale di quattro rilevazioni al giorno, invece che ogni 30 minuti come prevedeva precedentemente lo schema.

A scopo didattico abbiamo deciso di implementare una piccola pipeline Kafka in cui dividiamo la fase di raccolta dati (\textit{scraper\_producer.py}) e la fase di immagazzinamento (\textit{scraper\_consumer.py}). 

\begin{figure}[H]
	\centering
	\includegraphics[width=0.8\linewidth]{pics/pipeline.png}
	\caption{pipeline di raccolta dati - seconda sessione}
\end{figure}

Di seguito si può vedere nel dettaglio il procedimento dei due algoritmi:

\begin{algorithm}[H]
	\KwData{country - insieme dei paesi prescelti; videos - insieme di video scaricati da un determinato paese; KafkaProducer(data, channel) - sends data to channel kafka}
	\nl \For{every 6 hours} {
		\nl \ForEach{country}
		{
			\nl videos = APIrequest(max(50 video), country)\\
			\nl KafkaProducer(videos, yt\_video) \\ 
			\nl \While {tendency videos are finished}
			{
				\nl videos = APIrequest(max(50 video), country) \\
				\nl KafkaProducer(videos, yt\_video) \\
			}
		}
	}
	\caption{scraper producer}
\end{algorithm}

\begin{algorithm}[H]
	\KwData{KafkaConsumer(channel) - receives data from channel kafka}
	\nl \While{loop} {
		\nl \If{KafkaConsumer(yt\_video) $\ne \emptyset$}
		{
			\nl videos = KafkaConsumer(yt\_video)\\
			\nl videos\_fix = arrangeData(videos)\\ 
			\nl saveToJson(videos\_fix)\\
			\nl saveToMongoDB(videos\_fix)
		}
	}
	\caption{scraper consumer}
\end{algorithm}
I dati così raccolti sono stati salvati in formato json nella cartella [INSERIRE NOME CARTELLA].

\underline{Nota}: abbiamo scelto di salvare i dati in formato Json in modo che fosse facile caricarli in un nuovo server mongoDB e renderli più trasportabili. Per il caricamento vedi lo script: \textit{json\_to\_mongo.py}

Lo schema logico dei documenti json è sostanzialmente lo stesso di quello dei csv visto precedentemente in \textit{Tabella 1} ma con due variazioni:
\begin{itemize}
	\item \textbf{tags} immagazzinati come array di stringhe
	
	Ad esempio:
	
	tags: ["covid-19", "quarantena"]
	\item informazioni relative ai likes, dislikes, view\_count e comment\_count innestati in un oggetto \textbf{statistics}
	
	Ad esempio:
	
	statistics: \{view\_count: 14235,
				likes: 513,
				dislikes: 34,
				comment\_count: 254
				\}
\end{itemize}
\subsection*{Dati Covid-19}

I dati relativi ai Covid-19 sono stati estrapolati da  \href{https://ourworldindata.org/coronavirus-testing}{questo link.} I dati vengono estrapolati in formato \textit{csv}; dopo averli puliti essi contengono le seguenti informazioni:
\begin{table}[H]
	\centering
	\begin{tabular}{l|l}
		\textbf{Attributo} & \textbf{Descrizione} \\
		\hline
		\textbf{iso\_code} & codice unico riferito al paese \\\hline
		\textbf{location} & nome esteso del paese\\\hline
		\textbf{date} & data della rilevazione\\\hline
		\textbf{total\_cases} & numero totale dei casi\\\hline
		\textbf{new\_cases} & nuovi casi registrati in quella giornata.\\\hline

	\end{tabular}
	\caption{Schema degli attributi dei dati sul Covid-19.}
\end{table}


  
\subsection*{Qualità dati}
La grande disponibilità di dati ha rappresentato il problema più rilevante della verifica di qualità. Fortunatamente i dati non presentavano problemi di \textit{missing values}, che avrebbero costituito una difficoltà non trascurabile. Le principali problematiche emerse sono due:
\begin{itemize}
	
	\item Ridondanza: I dati di dicembre-gennaio presentano richieste effettuate ai server di Youtube ogni mezz'ora, a differenza del periodo successivo. Di conseguenza sono stati rilevati molti dati simili tra loro, senza alcuna sostanziale variazione nelle varie fasce delle giornata. Per risolvere il problema abbiamo optato per scegliere quattro rilevazioni distaccate di sei ore ciascuna, in maniera da uniformare i dati con quelli di marzo-maggio. Per ulteriori dettagli è possibile visualizzare il notebook jupyter con cui è stato affrontato il problema.

	\item Saturazione richieste: Google non permette di effettuare troppe richieste nella stessa giornata. Per arginare il problema abbiamo utilizzato tre API key differenti da alternare durante la presa dati. Nonostante questo espediente, ci sono stati alcuni momenti dove non è stato possibile effettuare le richieste ai server. Come risoluzione di questo problema abbiamo duplicato i dati della richiesta precedente, poiché le tendenze estratte in tempi ravvicinati non presentano variazioni importanti nei video.
	
	
\end{itemize}

\subsection*{Integrazione dati}
	Per poter rispondere alle nostre domande di ricerca abbiamo dovuto effettuare un'integrazione tra i dati dei video di Youtube e i dati relativi alla pandemia da Covid-19. L'integrazione è avvenuta prima del caricamento dei dati su MongoDB, lo script che mostra l'operazione è \textit{merge\_to\_mongo.py}. 
	
	Per ricavare le informazioni relative all'andamento della pandemia nella giornata considerata abbiamo effettuato un'\textbf{integrazione temporale}, dove le chiavi considerate sono state il \textbf{timestamp} e il \textbf{country\_name}, cioè la data e il paese. Il procedimento è il seguente: viene ricercata la data del video considerato e il paese di appartenenza, e successivamente viene creato un dizionario con tutte le informazioni della pandemia nella data e paese appena cercato. Infine, viene creata una nuova chiave \textit{covid} contenente un nuovo documento nidificato, che è il dizionario creato precedentemente. Solo a questo punto i documenti vengono caricati su MongoDB.
	
	Alcune date presentano un fuso orario che non ha alcun riscontro nel dataset covid, perché alcuni paesi presentano diversi fusi orari. Come risoluzione sono stati spostati tutti gli orari in base al fuso orario della capitale del paese di appartenenza del video. Questa correzione è stata effettuata all'interno dello script \textit{scraper\_consumer} direttamente, in un'ottica process driven. I fusi orari utilizzati sono presenti all'interno del file \textit{country\_name.json}.

	L'operazione di integrazione complessivamente richiede:
	\begin{itemize}
		\item senza sharding: 1419 s (23,6 min circa)
		\item con sharding: 1431 s (24 min circa)
	\end{itemize}

\subsection*{Espressione regolare}
	Per poter distinguere quali video possano essere considerati legati al Covid-19 o meno, abbiamo definito la seguente espressione regolare:
	\begin{figure}[H]
		\centering
		\includegraphics[width=0.95\linewidth]{pics/er.png}
	\end{figure}

	Come è possibile vedere, si è cercato di includere tutte le parole relative alla pandemia e la loro traduzione nelle lingue di tutti i paesi considerati.
	
	L'espressione regolare è stata applicata sia ai titoli dei video, sia ai tag scelti per descrivere il video. Di seguito le query applicate:
	
	\begin{enumerate}
		\item Vengono create due nuove variabili che identificano se nel video sono presenti riferimenti al Covid-19 o meno, una per il titolo e una per i tag. Questa variabile viene inizializzata come \textit{false}:
		
			db.video\_merge.update(\{\},\{\$set : \{covid\_tags : false, covid\_title : false\}\},\{multi : true\})
		
		\item I video vengono analizzati singolarmente e, se l'espressione regolare (REGEX) restituisce un match positivo, la variabile viene modificata in \textit{true}, prima per i tag:
		
			db.video\_merge.update(\{tags : \{\$in : [REGEX]\}\}, \{\$set : \{covid\_tags: true\}\}, \{multi : true\})
		
		\item Successivamente per il titolo:
		
			db.video\_merge.update(\{title : \{\$in : [REGEX]\}\}, \{\$set : \{covid\_title: true\}\}, \{multi : true\})
	\end{enumerate}
	L'applicazione delle query viene effettuata dallo script \textit{query\_covid.py}.
	
\subsection*{Scalabilità dell'algoritmo}

Una delle V su cui è stata posta la nostra attenzione è la Volume, ovvero come trattare e gestire grandi quantità di dati. 
In particolare abbiamo gestito:
\begin{itemize}
	\item prima sessione di scraping: 1.91 Gb
	\item seconda sessione di scraping: 1.11 Gb
	\item dati covid: 10 Mb
\end{itemize}
Per un totale di circa 3 Gb.

Abbiamo deciso di utilizzare per trattare i dati MongoDB, pertanto abbiamo implementato lo \textbf{Sharding}.
Sono stati costruiti tre shard, tutti in modalità replica set per garantire ridondanza in caso di guasti e frammentazione per rendere le query più efficienti nel momento in cui si andava a interrogare il \textit{router mongos}. I \textit{config server} sono stati anch'essi configurati come replica set. Come chiave di sharding è stato scelto il campo \textbf{country\_name}, perché le query per rispondere alle nostre domande vengono fatte sul singolo paese. Questo approccio è stato pensato anche nell'ottica di dividere la grande mole di dati in server disposti per ciascun paese. 

Per il dettaglio dei file di configurazione e altro si fa riferimento alla cartella \textit{sharding}.

Lo schema logico applicato è il seguente:
\begin{figure}[H]
	\centering
	\includegraphics[width=0.8\linewidth]{pics/sharding.png}
	\caption{Pipeline sharding}
\end{figure}

In definitiva viene creato un database mongo chiamato \textit{YT\_data} con due collection: 
\begin{itemize}
	\item \textit{video\_all}: contiene tutti dati relativi alle due sessioni di presa dati
	\item \textit{video\_covid}: contiene i dati relativi alla seconda presa dati integrati con quelli relativi al covid
\end{itemize}

\subsection*{Query}
	Abbiamo eseguito alcune query, per testare il funzionamento del database con e senza sharding, contenute negli script presenti nella cartella \textit{query}. In particolare si nota come la query 4, 5, 6 presentino dei risultati interessanti.
	
	In particolare:
	\begin{itemize}
		\item \textit{query 4}: elenco video a tema covid della categoria intrattenimento d'Italia
		\item \textit{query 5}: elenco video a tema covid della categoria intrattenimento e musica in Russia
		\item \textit{query 6}: elenco video a tema covid della categoria news \& politics in Italia e Germania
	\end{itemize}
	
	Per quanto riguarda tempo di esecuzione in millisecondi(\textit{executionTimeMillis}):
	\begin{table}[H]
		\centering
		\begin{tabular}{c|c|c}
			\textbf{Query} & \textbf{No Sharding} & \textbf{Sharding} \\
			\hline
			Query 4 & 635 & 126 \\
			Query 5 & 802 & 199 \\
			Query 6 & 1119 & 538 
		\end{tabular}
		\caption{differenze tempi in millisecondi}
	\end{table}
	
	Riguardo al numero di documenti esplorati per ottenere la risposta:
	\begin{table}[H]
		\centering
		\begin{tabular}{c|c|c}
			\textbf{Query} & \textbf{No Sharding} & \textbf{Sharding} \\
			\hline
			Query 4 & 444207 & 36950 \\
			Query 5 & 444207 & 28715 \\
			Query 6 & 444207 & 444207 
		\end{tabular}
		\caption{differenze numero di documenti esplorati}
	\end{table}
	
	Si può notare come per le query 4 e 5 vengono esplorati un numero di video molto inferiori nel database con sharding rispetto a quello senza, questo si traduce in un miglioramento dei tempi di esecuzione. La query 6, poiché necessità di informazioni riguardanti tutti i paesi, esplora tutti i dati a prescindere degli shard, quindi non ha alcun miglioramento prestazionale.
	
	Per il dettaglio vedere i file di risposta nella cartella \textit{query}.

\section*{Visualizzazione}

La scelta delle visualizzazioni è stata guidata dalle domande di ricerca che ci siamo posti: 
\begin{itemize}
	\item Come variano le tipologie dei video in tendenza dal periodo precedente al coronavirus alla quarantena?
	\item È vero che la fruizione di video su Youtube riguardanti il Covid-19 segue l'andamento dei dati sull'epidemia?
\end{itemize}

\subsection*{Scelta features}
Per rispondere in modo coerente alle nostre domande di ricerca abbiamo deciso di concentrarci sulle seguenti features del nostro dataset.
\begin{itemize}
	\item View Count
	\item Covid Title
	\item Covid Tags
	\item Trending Date
	\item Title
	\item Cases New
\end{itemize}

\subsection*{Scelta della visualizzazione}

Abbiamo utilizzato due infografiche diverse per le domande di ricerca, poiché ci è sembrata incompatibile un'unica visualizzazione per rispondere ad entrambe le domande di ricerca.


\paragraph{Prima infografica} La prima infografica consiste nella combinazione di due diverse visualizzazioni:
\begin{itemize}
	\item Un \textbf{lollipop chart} temporale che rappresenta il numero video entrato in tendenza ogni giorno. Sottolineamo che l'asse orizzontale rappresenta le date di dicembre-gennaio e di marzo-maggio, in accordo con la nostra raccolta dati.
	\item Un \textbf{bubble chart}, dove ogni bolla rappresenta un video, la sua grandezza rappresenta il numero di visualizzazioni e il colore la categoria di appartenenza.
\end{itemize}

La combinazione di queste due visualizzazioni permette di capire se il contenuto dei video entrati in tendenza nel periodo precedente al Covid-19 e durante la quarantena differiscano significativamente. Per farlo è sufficiente selezionare due giorni contemporaneamente e vedere il cambiamento nelle bolle. L'esplorazione di questa infografica avviene per passi guidati attraverso una storia, in modo da accompagnare l'utente attraverso tutte le informazioni che questa visualizzazione può offrire. Di seguito una visione sommaria dell'infografica comprensiva dei contesti:
\begin{figure}[H]
	\centering
	\includegraphics[height=0.5 \linewidth]{pics/prima_infografica.png}
	\caption{Prima infografica.}
\end{figure}

\paragraph{Seconda infografica} Per quanto riguarda la risposta alla seconda domanda di ricerca abbiamo deciso di utilizzare un'infografica composta da due visualizzazioni come precedentemente. In particolare:
\begin{itemize}
	\item Un misto tra un \textbf{bar chart} e un \textbf{line chart} temporale. In questa visualizzazione abbiamo fatto risaltare la differenza percentuale tra una rilevazione e quella del giorno precedente. Il line chart riguarda l'aumento percentuale del numero di video in tendenza relativi al Covid-19 mentre il bar chart riguarda l'aumento percentuale dei nuovi casi di Covid-19 rispetto al giorno precedente. 
	Il numero di video considerati come riguardanti il coronavirus sono quelli che presentano la feature \textit{covid-title} \textbf{o} \textit{covid-tags} a 1, cioè che mostrano nel titolo \textbf{oppure} nei tag parole inerenti la pandemia.
	Abbiamo utilizzato questo tipo di grafico in modo da poter vedere se le notizie negative o positive dei dati riguardanti il Covid-19 abbia influito sulla fruizione online dei video concernenti lo stesso argomento.
	L'utilizzo di un bar chart e un line chart è dovuto al fatto che dai questionari è risultata più apprezzata questa scelta per riconoscere le due variabili.

	\item Uno \textbf{stacked bar chart} che rappresenta il numero di video per categoria che riguardano il Covid-19. La visualizzazione può essere filtrata per giorno semplicemente interagendo con la prima visualizzazione.
\end{itemize}

La combinazione di queste due visualizzazioni consente di rispondere alla seconda domanda di ricerca che ci siamo posti precedentemente. Proponiamo di seguito una visione sommaria dell'infografica comprensiva dei contesti.

\begin{figure}[H]
	\centering
	\includegraphics[height=0.5 \linewidth]{pics/seconda_infografica.png}
	\caption{Seconda infografica.}
\end{figure}
\subsection*{Valutazione della qualità}
La valutazione della qualità si è articolata in tre passaggi:

\paragraph{User Test} Durante questa fase ci siamo occupati di sottoporre la nostra infografica a sei persone lasciando completa libertà di esplorazione. Le varie interazioni sono state registrate in modo da far sì che potessero emergere le diverse problematiche di cui non ci siamo accorti in fase di realizzazione delle infografiche. Esponiamo le problematiche emerse durante questa fase di valutazione e le correzioni applicate:

\begin{itemize}
	\item \textit{Problema 1:} Il fatto di avere due variabili sotto forma di linea nella prima infografica rende difficoltoso distinguerle nonostante il colore.\\\textit{Soluzione 1:} Abbiamo deciso di assegnare ad una variabile la forma "linea" e all'altra variabile la forma "barra".
	\item  \textit{Problema 2: Come comprendere che cliccare sul bianco significa non avere nessun giorno selezionato, e quindi una visione complessiva.}\\\textit{Soluzione 2: Introdurre l'infografica mediante storie che possano guidare l'utente nell'esplorazione.} 
	\item  \textit{Problema 3: Come comprendere quali video sono nella categoria Covid e quali no.} \\\textit{Soluzione 3: Una legenda semplificata e l'utilizzo di bar chart sovrapposti di colori diversi.} 
\end{itemize}
\paragraph{Risultati dei task} Durante questa fase ci siamo occupati di sottoporre tre diverse richieste per ogni infografica ad almeno 6 utenti; questi task devono essere soddisfatti esplorando in maniera interattiva. I task da risolvere sono stati i seguenti:
\begin{enumerate}
	\item Per la \textit{prima infografica:}
	\begin{itemize}
		\item Nella giornata del 20 marzo in USA quale video ha avuto più visualizzazioni e a quale categoria appartiene?  \textit{Entertainment, Eminem Godzilla}
		\item Della categoria Entertainment quali sono i canali che hanno fatto più visualizzazioni a Natale e a Pasqua in Germania? \textit{The Late Show with Corbin, Mr Beast}
		\item Nella categoria sport confronta il 30 dicembre e 10 aprile in Germania. Qual è il titolo dei video più visti per ciascun giorno?\\
		\textit{Sampdoria 1-2 Juventus | Ronaldo Header Winds it for the Visitors | Serie A TIM \\ F1 Esports Virtual Grand Prix Highlights | Aramco}
	\end{itemize}
	\item Per la \textit{seconda infografica:}
\begin{itemize}
	\item Trova la categoria che ha avuto più video Covid-19 il giorno 27 Marzo in Italia? \textit{News and Politics}
	\item Quanti video Covid-19 della categoria "People and Blogs" ci sono stati negli USA? \textit{12}
	\item Quanti nuovi contagi ha avuto la Corea del Sud il 12 Aprile? \textit{62}
\end{itemize}
\end{enumerate}

Sono stati registrati i tempi in cui gli utenti riuscivano a completare questi obiettivi e sono stati visualizzati i risultati nei seguenti box plot:
\begin{figure}[H]
	\centering
	\includegraphics[height=0.5 \linewidth]{../quality/tempi_box_plot_seaborn.png}
	\caption{Tempi di completamento dei task.}
\end{figure}
Questa visualizzazione è utile per capire se le nostre infografiche sono troppo dispersive oppure se riescono a essere facili e intuitive. Come si nota l'esecuzione del primo della seconda infografica è quella che permette di andare 

\paragraph{Questionari} Per quanto riguarda l'ultima fase è stato somministrato un questionario di valutazione della qualità a 24 persone articolato nella seguente maniera:
\begin{itemize}
	\item Come valuti la chiarezza dell' infografica?
	\item Come valuti l'utilità dell'infografica?
	\item Quanto valuti la bellezza dell'infografica?
	\item Come valuti l'intuitività dell'infografica?
	\item Quanto è stata informativa l'infografica?
	\item Come valuti complessivamente l'infografica?
\end{itemize}
Le risposte sono state registrate grazie al tool "Moduli di Google". Una volta registrati i risultati è stata controllata che la valutazione dell'infografica fosse coerente con una ricostruzione complessiva della regressione con i coefficienti di Cabitza-Locoro.
\begin{figure}[H]
   \includegraphics[width=0.475\textwidth]{../quality/risposte_violin_plot_first.png}
   \hfill
   \includegraphics[width=0.475\textwidth]{../quality/risposte_violin_plot_second.png}
\end{figure}
L'utilizzo di un box plot è risultato molto comodo per la registrazione delle risposte poiché la media è un indicatore di tendenza centrale e non fornisce alcuna informazione sulla distribuzione di questi dati.
Per quanto riguarda invece la coerenza della valutazione rispetto alla ricostruzione data dai coefficienti abbiamo avuto la seguente distribuzione:
\begin{figure}[H]
   \includegraphics[width=0.475\textwidth]{../quality/risposte_scatter_plot_first.png}
   \hfill
   \includegraphics[width=0.475\textwidth]{../quality/risposte_scatter_plot_second.png}
   \caption{$R_1^2 = 0.79, R_2^2 = 0.73$}
\end{figure}

Ci siamo inoltre preoccupati di calcolare le correlazioni tra le variabili sottoposte nei questionari. Possiamo quindi notare che l'intuitività è la caratteristica che influisce meno delle altre sulla valutazione complessiva della nostra infografica.
\begin{figure}[H]
   \includegraphics[width=0.475\textwidth]{../quality/risposte_correlation_plot_first.png}
   \hfill
   \includegraphics[width=0.475\textwidth]{../quality/risposte_correlation_plot_second.png}
\end{figure}
\section*{Conclusioni e prospettive future}
Visti i risultati di questo lavoro possiamo affermare che vi sono rilevanti differenze nelle tendenze di Youtube prima e durante l'epidemia di Covid-19. 

La prima infografica mostra come la distribuzione dei video nelle varie categorie sia cambiata. Ad esempio, per quanto riguarda l'Italia la categoria Sport subisce una drastica diminuzione, presumibilmente dovuta al blocco di tutte le attività sportive durante la quarantena; così come la categoria News \& politics che risulta essere più presente nel periodo Covid rispetto al precedente.

\textit{La seconda infografica mostra che la variazione giornaliera del numero di contagi non influenza strettamente la quantità di contenuti nelle tendenze legati al Covid.} In alcuni frangenti si nota una certa relazione tra le due misure, ad esempio, in Italia nel periodo di fine marzo e inizio aprile. Ne consegue che, probabilmente, l'impatto dei numeri ufficiali relativi ai contagi non abbiano influenzato in modo significativo le tendenze giornaliere di Youtube.

Da queste due infografiche possiamo concludere che i gusti degli utenti di Youtube hanno subito un evidente cambiamento durante il periodo dell'epidemia, e che questo mutamento è avvenuto gradualmente e non è necessariamente legato al numero ufficiale dei contagi.

\paragraph{Prospettive future\\}

Un aspetto che può essere interessante approfondire riguarda il legame che esiste tra la variazione giornaliera di contagi e il numero di contenuti Covid presenti sulla piattaforma. Un'analisi statistica attraverso regressioni semplici o complesse potrebbe rispondere alla seconda domanda di ricerca in maniera più completa ed esaustiva. Potrebbero essere considerati non solo i nuovi contagi giornalieri, ma anche altri fattori, come il numero di decessi o la quantità di contagi totale nel paese.
Ulteriori analisi potrebbero essere svolte su altre piattaforme, per avere un quadro più generale dei movimenti online che la pandemia ha generato. In tal modo sarebbe possibile comprendere come i gusti delle persone siano cambiati, o come il tempo passato in casa abbia cambiato gli interessi.

Un'ultima considerazione andrebbe fatta riguardo al periodo successivo alla fase 2. In questo modo si potrebbe verificare se i cambiamenti che sono avvenuti abbiano davvero cambiato le tipologie di video più usufruite oppure se questa tendenza a spostarsi verso video dai contenuti più informativi si tratti solamente di un periodo transitorio.

\end{document}