\subsection*{Scelta degli strumenti}
Per rispondere alle nostre domande di ricerca dobbiamo comprendere su quali delle tre "V" della Data Management avremmo dovuto concentrare i nostri sforzi. La conclusione a cui siamo giunti si focalizza sulla \textit{Volume} e \textit{Variety}. Il trattamento di grandi quantità di dati è risultato fondamentale, in quanto tramite le API di Youtube è stato possibile ricavare una discreta quantità di informazioni riguardanti i video in tendenza ($\sim 3Gb$). I dati riguardanti la pandemia sono stati ricavati in formato \textit{csv}, di conseguenza l'integrazione con i dati di Youtube, in formato \textit{json}, ha rappresentato una parte importante.

Per la gestione dei dati da Youtube ci siamo affidati al software MongoDB, sul quale sono stati caricati i dati mediante script Python. Le API di Youtube non permettevano di scegliere il periodo, ma fornivano i dati in tempo reale, per questo abbiamo utilizzato Apache Kafka per salvare i dati e poi poterli caricare su MongoDB.
 
Per poter lavorare collettivamente e tener traccia di tutti i passi del progetto abbiamo utilizzato \textbf{Git Hub} come software di controllo versioni al link \href{https://github.com/fedelux3/Youtube_trending_analysis}{GitHub} 
