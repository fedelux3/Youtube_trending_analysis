\documentclass[10pt, a4paper,openany]{article}
\usepackage[italian]{babel}
\usepackage[T1]{fontenc}
\usepackage[table]{xcolor}
\usepackage{float}
\restylefloat{table,figure}
\usepackage{graphicx}	
\usepackage[utf8]{inputenc}
\usepackage{amsmath}
\usepackage{fancyhdr}
\usepackage{geometry}
\geometry{a4paper,top=2cm,bottom=2cm,left=2.5cm,right=2.5cm,%
	    heightrounded,bindingoffset=5mm}
\usepackage{amssymb}
\usepackage{amsthm}
\usepackage{multicol}
\usepackage{xcolor}
\usepackage{hyperref}
\usepackage{caption}
%\usepackage{url}
%inizia qui aggiunta
\usepackage{collcell}
\usepackage{hhline}
\usepackage{pgf}
\usepackage{multirow}
 \usepackage{fancyvrb}
\usepackage{verbatim}


\begin{document}
\begin{center}
\large\textbf{Guida operativa}
\end{center}

\begin{center}
 Federico Luzzi,  Marco Peracchi, Christian Uccheddu, Gabriele Centemeri (TTC)
\end{center}
Di seguito la guida operativa per eseguire il codice in modo da replicare i risultati ottenuti:

\section*{Presa dati Dicembre}

Eseguire il file:

\begin{verbatim}
scraper/scraper_csv.py
\end{verbatim}
Questo script serve ad effettuare una rilevazione dati ogni 6h. I dati così estratti vengono trasformati in csv.
Trasformare i dati da csv in json usando: 

\begin{verbatim}
csv_to_json/main.py
\end{verbatim}
Per eseguire lo script è necessario fornire il seguente parametro:
\begin{itemize}
	\item -d "directory\_dei\_dati"
\end{itemize}
Caricare quindi i json su mongo usando: 

\begin{verbatim}
json_to_mongo/json_to_mongo.py
\end{verbatim}
A questo script vanno forniti i seguenti parametri:

\begin{itemize}
	\item -d "directory dei dati"
	\item -u "utente mongo"
	\item -p "password utente"
	\item -port "porta in cui è attivo l'utente"
	\item-db "Nome del database in output"
	\item -c "collection in cui vengono inseriti i dati"
\end{itemize}

\section*{Presa dati periodo Covid}

Aprire il servizio mongo da terminale.


Lanciare in due terminali contemporaneamente: 

\begin{itemize}
	\item \begin{verbatim}
		scraper/scraper_consumer.py
	\end{verbatim}
	\item \begin{verbatim}
	 scraper/scraper_producer.py
	\end{verbatim}
\end{itemize}

Questo effettua una rilevazione dati ogni 6h attraverso il servizio kafka. L'utilizzo di kafka non è indispensabile, inizialmente però avevamo deciso di prendere i dati sia dei canali che dei video in live stream quindi kafka aveva senso in quanto venivano usati due topic diversi e fatte le prime operazioni preliminari. Abbiamo deciso di tenerlo per non stravolgere la pipeline di esecuzione.

\section*{Presa dati Covid}

Scaricare i dati in formato csv da \href{https://ourworldindata.org/coronavirus-testing}{questo sito}
\\Eseguire il codice:

\begin{verbatim}
covid/cleaner.py
\end{verbatim}
Questo script permette di eseguire una pulizia dei dati in modo da renderli integrabili con i json raccolti in precedenza.


\section*{Integrazione dei dati}


Per eseguire l'integrazione tra i dati covid e i dati di youtube bisogna eseguire il seguente script:

\begin{verbatim}
clean_store_data/merge_to_mongo.py
\end{verbatim}
A questo script vanno forniti i seguenti parametri:

\begin{itemize}
	\item -d "directory dei dati"
	\item -u "utente mongo"
	\item -p "password utente"
	\item -port "porta in cui è attivo l'utente"
	\item-db "Nome del database in output"
	\item -c "collection in cui vengono inseriti i dati"
\end{itemize}
Questo script integra i due dataset e carica tutto su mongo.


\section*{Query mongo}


Per le visualizzazioni che intendiamo fare abbiamo bisogno di poter distinguere quando un video contenga nel titolo o nei tag una delle parole che si rifanno al coronavirus. Per controllare questo è stata costruita l'espressione regolare che si può trovare nel file:
\begin{verbatim}
README.md
\end{verbatim} 



%/(corona|covid|virus|pandemi[aec]|epidemi[aec]|tampon[ei]*|sierologico|mascherin[ae]|코로나 바이러스|fase\s*(2|due)|iorestoacasa|stayathome|lockdown|[qc]uar[ae]nt[äae]i*n[ea]|कोरोनावाइरस|ਕੋਰੋਨਾਵਾਇਰਸ|massisolation|distanziamento\s*sociale|social\s*distancing|감염병 세계적 유행|パンデミック|コロナウイルス|सर्वव्यापी महामारी|ਸਰਬਵਿਆਪੀ ਮਹਾਂਮਾਰੀ|пандемия|коронавирус|social\s*distancing|distanciamiento\s*social|코로나|कोविड|ਕੋਵਿਡ|vaccin[oe]*|isolamento|intensiv[ao]|assembrament[io]|guant[oi]|dpi|disinfettante|swabs|emergenza|emergency|droplets*|aerosol|isolation|intensive\s*care|crowd|gloves*|disinfectant|감염병 유행|완충기|마스크|나는 집에있어|폐쇄|사회적 거리두기|백신|모임|비상 사태|비말|범 혈증|écouvillon|masques*|restealamaison|confin[ae]mento*|distanciation\s*sociale|soins\s*intensifs|rassemblements|désinfectant|urgence|gouttelettes|飛沫|タンポン|マスケリン|封鎖|人混みを避ける|ワクチン|隔離|集会|集中治療|緊急|बूंदें|फाहे|मास्क|लॉकडाउन|सोशल डिस्टन्सिंग|टीका|गहन देखभाल|समारोहों|आपातकालीन|gotas|cotonetes|m[áa]scaras|ficoemcasa|vac[iu]na|reuni[õo]n*es|emerg[êe]ncia|капли|тампоны|маски|карантин|социальное\s*дистанцирование|вакцина|интенсивная\s*терапия|сходы|чрезвычайное\s*происшествие|hisopos|mequedoencasa|cierre|Tröpfchen|Tupfer|Masken|bleibezuHause|Ausgangssperre|soziale\s*Distanzierung|Impfstoff|Intensivstation|Versammlungen|Notfall|건강\s*격리|検疫|संगरोध|[кК]арантин)/i


Creare due nuovi campi chiamati \textbf{covid\_title} e \textbf{covid\_tags} settati entrambi a \textbf{False} 

\begin{Verbatim}[frame=single,baselinestretch=0.1]
db.video_merge.update({},{$set : {covid_tags : false, 
				covid_title : false}},
				{multi : true})
\end{Verbatim}
Eseguire le seguenti due query che controllano se l'espressione regolare è presente nel campo title o in uno dei tag per ogni video.

\begin{Verbatim}[frame=single,baselinestretch=0.1]
db.video_merge.update({tags : {$in : [REGEX]}}, 
			{$set : {covid_tags: true}}, 
			{multi : true})
\end{Verbatim}
 \begin{Verbatim}[frame=single,baselinestretch=0.1]
db.video_merge.update({title : {$in : [REGEX]}}, 
			{$set : {covid_title: true}}, 
			{multi : true})
\end{Verbatim}


\end{document}


