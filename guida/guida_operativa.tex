\documentclass[10pt, a4paper,openany]{article}
\usepackage[italian]{babel}
\usepackage[T1]{fontenc}
\usepackage[table]{xcolor}
\usepackage{float}
\restylefloat{table,figure}
\usepackage{graphicx}	
\usepackage[utf8]{inputenc}
\usepackage{amsmath}
\usepackage{fancyhdr}
\usepackage{geometry}
\geometry{a4paper,top=2cm,bottom=2cm,left=2.5cm,right=2.5cm,%
	    heightrounded,bindingoffset=5mm}
\usepackage{amssymb}
\usepackage{amsthm}
\usepackage{multicol}
\usepackage{xcolor}
\usepackage{hyperref}
\usepackage{caption}
%\usepackage{url}
%inizia qui aggiunta
\usepackage{collcell}
\usepackage{hhline}
\usepackage{pgf}
\usepackage{multirow}
\usepackage{fancyvrb}
\usepackage{verbatim}


\begin{document}
\begin{center}
\large\textbf{Guida operativa}
\end{center}

\begin{center}
 Federico Luzzi,  Marco Peracchi, Christian Uccheddu, Gabriele Centemeri (TTC)
\end{center}
Di seguito la guida operativa per eseguire il codice e replicare i risultati ottenuti:

\section*{Presa dati Dicembre-Gennaio}

Eseguire il file che esegue richieste ogni 30 minuti e salva i dati in formato csv. 

\begin{verbatim}
scraper/scraper_csv.py
\end{verbatim}

Il jupyter-notebook serve a selezionare prese dati ogni 6 ore, e si trova all'interno della cartella \textit{clean\_store\_data}.

\begin{verbatim}
clean_jen_video/fix_jen_json.ipynb
\end{verbatim}

Per trasformare i dati da formato csv a json.

\begin{verbatim}
csv_to_json/main.py
\end{verbatim}

Lo script ha bisogno del parametro:

\begin{itemize}
	\item -d "directory\_dei\_dati"
\end{itemize}

In tal modo si indica la directory dove sono i dati da trasformare.
Per caricare i json su MongoDB avviare lo script:

\begin{verbatim}
json_to_mongo(windows)/json_to_mongo.py
\end{verbatim}

A questo script vanno forniti i seguenti parametri:

\begin{itemize}
	\item -d "directory dei dati"
	\item -u "utente mongo"
	\item -p "password utente"
	\item -port "porta in cui è attivo l'utente"
	\item-db "Nome del database in output"
	\item -c "collection in cui vengono inseriti i dati"
\end{itemize}


\section*{Presa dati Marzo-Maggio}

Aprire il servizio Mongo da terminale e successivamente avviare in due terminali separati gli script:

\begin{itemize}
	\item \begin{verbatim}
	scraper/scraper_consumer.py
	\end{verbatim}
	\item \begin{verbatim}
	scraper/scraper_producer.py
	\end{verbatim}
\end{itemize}

Mediante l'utilizzo del servizio Kafka vengono rilevati i dati ogni 6 ore.

\section*{Presa dati Covid}

Scaricare i dati in formato csv da \href{https://ourworldindata.org/coronavirus-testing}{OurWorldInData}
ed eseguire il codice:

\begin{verbatim}
covid/cleaner.py
\end{verbatim}

Questo script permette di eseguire una pulizia dei dati in modo da renderli integrabili con i json dei video precedenti.

\section*{Integrazione dei dati}


Per eseguire l'integrazione tra i dati Covid e i dati di Youtube bisogna eseguire il seguente script:

\begin{verbatim}
json_to_mongo(windows)/merge_to_mongo.py
\end{verbatim}

A questo script vanno forniti i seguenti parametri:

\begin{itemize}
	\item -d "directory dei dati"
	\item -u "utente mongo"
	\item -p "password utente"
	\item -port "porta in cui è attivo l'utente"
	\item-db "Nome del database in output"
	\item -c "collection in cui vengono inseriti i dati"
\end{itemize}

I dati vengono quindi integrati sulla data e il paese e successivamente caricati su Mongo.

\section*{Query mongo}

La seconda domanda di ricerca ci chiede di distinguere quali video riguardino il coronavirus o meno. L'espressione regolare seguente analizza tags e titoli per verificare se è presente una parola riferita alla pandemia.

\begin{figure}[H]
	\centering
	\includegraphics[height=0.4 \linewidth]{er.png}
	\caption{Regular expression usata}
\end{figure}

Eseguire ora i seguenti comandi all'interno della shell di MongoDB.

Creare due nuovi campi chiamati \textbf{covid\_title} e \textbf{covid\_tags} inizializzati entrambi a \textbf{False} 

\begin{Verbatim}[frame=single,baselinestretch=0.1]
db.video_merge.update({},{$set : {covid_tags : false, 
			covid_title : false}},
			{multi : true})
\end{Verbatim}
Eseguire le seguenti due query che controllano se l'espressione regolare è presente nel campo title o in uno dei tag per ogni video.

\begin{Verbatim}[frame=single,baselinestretch=0.1]
db.video_merge.update({tags : {$in : [REGEX]}}, 
			{$set : {covid_tags: true}}, 
			{multi : true})
\end{Verbatim}
 \begin{Verbatim}[frame=single,baselinestretch=0.1]
db.video_merge.update({title : {$in : [REGEX]}}, 
			{$set : {covid_title: true}}, 
			{multi : true})
\end{Verbatim}

\section*{Sharding}

Lo sharding dei documenti di Mongo è stato eseguito in locale, quindi utilizzando \textit{localhost} come host. All'interno della cartella \textit{sharding} è possibile visualizzare le cartelle contenenti i vari file di configurazione per tutti i componenti.

\begin{itemize}
	\item \textbf{configsvr} Sono tre istanze di \textit{mongod}, configurate come replica set. Il \textit{config server} conosce dove ogni dato è allocato dei vari shard, quindi è importante configurarlo come replica-set, così che in caso di guasti non si perdano le informazioni.
	\item \textbf{router} \`E un'istanza di \textit{mongos}. Per interrogare i vari shard è necessario interfacciarsi con essa.
	\item \textbf{shard} Sono tre istanze di \textit{mongod}. Ogni shard è configurato in replica-set, e i dati vengono suddivisi nei vari shard.
\end{itemize}

All'interno di ogni cartella sono presenti i vari file di configurazione, dove deve essere specificato il percorso della cartella \textit{data} di ogni istanza. Per prima cosa inizializzare i replica-set, e successivamente avviarli come shard server o come config server. Una volta collegati quest'ultimi al router è necessario caricare i dati su uno degli shard. Prima di effettuare la procedura va specificata la \textit{shard key}, e va anche specificato il metodo di suddivisione dei dati, se \textit{hashed} o \textit{range}. In questo lavoro è stato utilizzato il primo metodo.






\end{document}


